\documentclass[12pt,a4paper]{report}

\usepackage{makrok}
\usepackage{listingsutf8}
\usepackage{color}

\author{Sós Nikolett\\Mérnök Informatikus BSc, 2. évfolyam\\\ \\Széchenyi István Egyetem, GIVK, Informatika Tanszék}
\title{Pénzérme keresés \\python programozással}

\definecolor{mygreen}{rgb}{0, 0.5, 0}
\lstset{ 
    backgroundcolor=\color{white},   
    basicstyle=\footnotesize,        % the size of the fonts that are used for the code
    breakatwhitespace=false,         % sets if automatic breaks should only happen at whitespace
    breaklines=true,                 % sets automatic line breaking
    captionpos=b,                    % sets the caption-position to bottom
    extendedchars=true,              
    frame=single,	                   % adds a frame around the code
    keepspaces=true,                 
    morekeywords={constraint, enum, set, where, if, output, array, of, var, in},
    keywordstyle=\color{mygreen},       % keyword style
    keywordstyle={[2]\color{blue}},
    keywords=[2]{forall, count, show, format, alldifferent, fix},
    numbers=left,                    
    numbersep=5pt,                   % how far the line-numbers are from the code
    rulecolor=\color{black},         
    showspaces=false,                
    showstringspaces=false,          % underline spaces within strings only
    showtabs=false,                  % show tabs within strings adding particular underscores
    stepnumber=1,                    
    tabsize=3,	                   % sets default tabsize to 2 spaces
}

\begin{document}

\maketitle


\chapter*{Kivonat}

    Az általam választott feladat pénzérmék felismerése és megtalálása volt egy megadott fényképen.

    A megoldó algoritmus python nyelven lett megvalósítva, opencv, mattplotlib és numpy könyvtárak kiegészítésével.

    Munkám során foglalkoztam mintakereséssel és foltdetektáló algoritmus létrehozásán is. 

    \textbf{Kulcsszavak}: pénzérme keresés, python programozás, mintakeresés, foltdetektálás


\tableofcontents


\chapter{Bevezetés}

    Beadandóm során a pénzérme számláló program megoldásával foglalkoztam.
    A feladvány megoldásához használt python nyelvet és a hozzá használt könyvtárakat az 1.1. fejezetben mutatom be.
    A megoldott feladat bemutatása az 1.2. alfejezetben olvasható.
    A 2. fejezetben a programom első felét, azaz a mintakeresést fogom bemutatni.
    A 3. fejezetben pedig a feladatom fő része, a foltdetektálás található meg.

\section{Python programozás bemutatása} 
	A pythonról, mint programozási nyelvről az 1980-as évek végétől beszélhetünk. A fejlesztést Guido van Rossum kezdte meg 1989-ben az amszterdami Matematikai és Informatikai Központban, a CWI kutatóközpontban. A név öt brit színésztől származik, akik a csoportjukat hívták így (Monty Python-csoport, közkeletű nevén The Pythons), valamint a több részes műsoruk címében is szerepelt a python szó. Az első verzió a 0.9.0.-es 1991 februárjában került a közönség elé, 1994-ben jelent meg az 1.0 verzió, a 2.0-ás, és végül a 3.0-ás verzió (amit én is használok) 2008-ban látott napvilágot\cite{python}.
	
	A python magasszintű programozási nyelvként szerepel a köztudatban, azonban itt a futási gyorsaság helyett a programozó munkájának megkönnyítése kerül előtérbe. 
	
\subsection{Opencv és numpy könyvtárak}
	Programom működéséhez két könyvtárat szükséges meghívni, az opencv-t és a numpy-t.
	 
	Az opencv kimondottan a számítógépes, azaz a gépi látásra kell kitalálva, megkönnyítve a vizuális elemek feldogozását, értelmezését és felhasználását. Ez a könyvtár python-on kívül más magas szintű programozási nyelvekben is megtalálható, mint például a C++-ban is\cite{opencv}.
	
	A numpy python nyelvre lett kifejlesztve, hogy a programban numerikus számításokat is végre tudjunk hajtani, tehát a több dimenziós tömbök és mátrixok használatát teszi lehetővé. Programomban a képek vannak pixelenként tárolva egy két dimenziós tömbben, így számos művelet végrehajtása lehetséges egyszerű módszerekkel\cite{numpy}.
	
	
\section{Feladvány bemutatása}
	
	A beadandóm témája a következő volt:
	Pénzérme számláló alkalmazás: Egy bekért képen megállapítani, hogy hány pénzérme található meg rajta. 
   

\chapter{Feladat megoldása}
	
	A programom elején meghívom a használt könyvtárakat és opencv-vel beolvastatok két képet. Az egyik kép amelyiken össze akarjuk számolni az érméket, a másik képre a mintakeresésnél fogok részletesebben kitérni.
	
\section{Mintakeresés} 

  	A beadandóm további része két fő részből áll. Az első, ahogy a címben is áll az a minta keresés. Ebben a részben lesz jelentősége a már korábban is megemlített második képnek. Ezen egy darab érme látható melynek jellemző pontjait szeretném megtalálni a másik vizsgált képen is, melyen több érme is van. Ezt egyszerű Bruce-Force algoritmussal hajtom végre. Ez a programrész úgy dolgozik, hogy mindkét képen megkeresi a jellemző pontokat és ha egyezést talál, akkor összeköti őket, majd kimenetben ábrázolja ezeket a kapcsolatokat. Jelenleg az első 10 találatot jeleníti meg az algoritmus. 

\section{Foltdetektálás} 
	
	A második fő rész a pénzérmék megszámlálása. Ezt az eredeti kép átszerkesztett verzióján hajtom végre. Gauss szűrővel a zajokat megszüntetem, majd "Otsu treshold-al" az egész képet két színnel jelenítem meg. Az egyik szín a háttér, a másik pedig a pénzérmék színe. Ezek után eróziót és dilatációt is használok az esetleges kép hibák kijavítására.
	A következő lépés a "SimpleBlobDetector" paramétereinek beállítása, majd ezekkel a konkrét foltdetektálás végrehajtása. Az érzékelt foltokat a kimeneten pirossal bekarikázza a program, majd pedig "len" függvénnyel meg is kapjuk ezeknek a számát amiről e-mailben is értesítést kapunk. 
      



\chapter{Összefoglalás}

	Összefoglalva munkám során megismerkedtem a python programozói nyelvvel és a hozzá használt könyvtárakkal, illetve  mintakereső és foltdetektáló algoritmusok létrehozásával foglalkoztam.


\addcontentsline{toc}{chapter}{Irodalomjegyzék}
\bibliographystyle{unsrt}
\bibliography{gepi_latas}

\end{document}